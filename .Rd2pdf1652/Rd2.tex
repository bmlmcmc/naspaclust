\documentclass[a4paper]{book}
\usepackage[times,inconsolata,hyper]{Rd}
\usepackage{makeidx}
\usepackage[utf8,latin1]{inputenc}
% \usepackage{graphicx} % @USE GRAPHICX@
\makeindex{}
\begin{document}
\chapter*{}
\begin{center}
{\textbf{\huge \R{} documentation}} \par\bigskip{{\Large of \file{man/fgwc.Rd}}}
\par\bigskip{\large \today}
\end{center}
\inputencoding{utf8}
\HeaderA{fgwc}{Fuzzy Geographicaly Weighted Clustering}{fgwc}
%
\begin{Description}\relax
Fuzzy clustering with addition of spatial configuration of membership matrix
\end{Description}
%
\begin{Usage}
\begin{verbatim}
fgwc(data, pop, distmat, algorithm = "classic", fgwc_param, opt_param)
\end{verbatim}
\end{Usage}
%
\begin{Arguments}
\begin{ldescription}
\item[\code{data}] an object of data with d>1. Can be \code{matrix} or \code{data.frame}. If your data is univariate, bind it with \code{1} to get a 2 columns.

\item[\code{pop}] an n*1 vector contains population.

\item[\code{distmat}] an n*n distance matrix between regions.

\item[\code{algorithm}] algorithm used for FGWC

\item[\code{fgwc\_param}] a vector that consists of FGWC parameter (see \code{\LinkA{fgwcuv}{fgwcuv}} for parameter details)

\item[\code{opt\_param}] a vector that consists of optimization algorithm parameter (see \code{\LinkA{fgwcuv}{fgwcuv}} for parameter details)
\end{ldescription}
\end{Arguments}
%
\begin{Details}\relax
Fuzzy Geographically Weighted Clustering (FGWC) was developed by Mason and Jacobson (2007) by adding 
neighborhood effects and population to configure the membership matrix in Fuzzy C-Means. There are several algorithms currently
provided in this package. The optimization algorithm uses the centroid as the parameter to be optimized. Here are the
algorithm that can be used.
\begin{itemize}

\item{} \code{"classic"} - The classical algorithm of FGWC based on Mason and Jacobson (2007) and Runkler and Katz (for membership optimization).
\item{} \code{"abc"} - Optimization using artificial bee colony algorithm based on Karaboga and Basturk (2010).
\item{} \code{"fpa"} - Optimization using flower pollination algorithm based on Yang (2012).
\item{} \code{"gsa"} - Optimization using gravitational search algorithm based on Rashedi (2009) and Li and Dong (2017).
\item{} \code{"hho"} - Optimization using harris-hawk optimization based on Bairathi (2018) and Heidari (2019). 
\item{} \code{"ifa"} - Optimization using intelligent firefly algorithm based on Yang (2012) and Fateen and Bonilla-Petriciolet (2014).
\item{} \code{"pso"} - Optimization using particle swarm optimization based on Kennedy and Eberhart (1996).
\item{} \code{"tlbo"} - Optimization using teaching-learning based optimization based on Rao et al.(2012) and Rao and Patel (2012).

\end{itemize}

the default parameter of FGWC (in case you do not want to tune anything) is \\{} \code{
c(kind='u',ncluster=2,m=2,distance='euclidean',order=2,alpha=0.7,a=1,b=1,max.iter=500,error=1e-5,randomN=1)}.\\{}
There is also a universal parameter to the optimization algorithm as well as the details. The default parameter
for the optimization algorithm is \\{}
\code{c(vi.dist='uniform',npar=10,par.no=2,par.dist='euclidean',par.order=2,pso=TRUE,
same=10,type='sim.annealing',ei.distr='normal',vmax=0.7,wmax=0.9,wmin=0.4,
chaos=4,x0='F',map=0.7,ind=1,skew=0,sca=1)} \\{}
If you do not define a certain parameter, the parameter will be set to its default value
\end{Details}
%
\begin{Value}
an object of class \code{"fgwc"}.\\{}
An \code{"fgwc"} object contains as follows:
\begin{itemize}

\item{} \code{converg} - the process convergence of objective function
\item{} \code{f\_obj} - objective function value
\item{} \code{membership} - membership matrix
\item{} \code{centroid} - centroid matrix
\item{} \code{validation} - validation indices (there are partition coefficient (\code{PC}), classification entropy (\code{CE}), 
SC index (\code{SC}), separation index (\code{SI}), Xie and Beni's index (\code{XB}), IFV index (\code{IFV}), and Kwon index (\code{Kwon}))
\item{} \code{max.iter} - Maximum iteration
\item{} \code{cluster} - the cluster of the data
\item{} \code{finaldata} - The final data (with the cluster)
\item{} \code{call} - the syntax called previously
\item{} \code{time} - computational time.

\end{itemize}

\end{Value}
%
\begin{SeeAlso}\relax
\code{\LinkA{fgwcuv}{fgwcuv}},\code{\LinkA{abcfgwc}{abcfgwc}},\code{\LinkA{fpafgwc}{fpafgwc}},
\code{\LinkA{gsafgwc}{gsafgwc}},\code{\LinkA{hhofgwc}{hhofgwc}},\code{\LinkA{ifafgwc}{ifafgwc}},\code{\LinkA{psofgwc}{psofgwc}},\code{\LinkA{tlbofgwc}{tlbofgwc}}
\end{SeeAlso}
%
\begin{Examples}
\begin{ExampleCode}
data('census2010')
data('census2010dist')
data('census2010pop')
# initiate parameter
param_fgwc <- c(kind='v',ncluster=3,m=2,distance='minkowski',order=3,
               alpha=0.5,a=1.2,b=1.2,max.iter=1000,error=1e-6,randomN=10)
## FGWC with classical algorithm
res1 = fgwc(census2010,census2010pop,census2010dist,'classic',param_fgwc,1)
## tune the ABC parameter
abc_param <- c(vi.dist='normal',npar=5,pso=FALSE,same=15,n.onlooker=5,limit=5) 
## FGWC with ABC optimization algorithm
res2 = fgwc(census2010,census2010pop,census2010dist,'abc',param_fgwc,abc_param) 
\end{ExampleCode}
\end{Examples}
\printindex{}
\end{document}
